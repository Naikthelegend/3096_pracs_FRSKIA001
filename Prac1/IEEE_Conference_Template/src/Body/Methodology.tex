\section{Methodology}

\subsection{Hardware}
The hardware used for this practical was a macOS laptop running Virtual Box with 4 CPU cores and 8 Gb of RAM. The Virtual Box was running Ubuntu image, which was in turn simulating a Raspberry Pi emulator. The Raspberry Pi emulator was used for the tests.

\subsection{Implementation}
The following code is the sudo code format of the two files: the python code which was used as the benchmark for the C code tests and the non-threaded C code.
The third file that was used in this test was a multi-threaded version of the C code file.
The \text{'data'} and \text{'carrier'} arrays are predefined arrays found in another tile

%Sudo
\begin{Cpp}
 # Load data arrays from external file
 Load data from external file

 # Define values
 c = carrier array
 d = data array
 result = empty array

 # Main function
 function main():
  Display "There are " + length of c + " samples"
  Display "using type " + type of first element in data array
  Call Timing.startlog()
  for i from 0 to length of c - 1:
  Append c[i] * d[i] to result array
  Call Timing.endlog()
\end{Cpp}

\subsection{Experiment Procedure}
The experiment start by benchmarking in Python using the following code. All tests were run 5 times and then an average was taken.

\begin{Cpp}
 cd Python
 for i in {1..5}; do python3 PythonHeterodyning.py; done
\end{Cpp}

The C code was tested next starting with the non-threaded version using the following code.
\begin{Cpp}
 cd ../C
 make
 for i in {1..5}; do make run; done
\end{Cpp}

The threaded version of the C code was tested next, where each test changed the defined number of threaded in the CHeterodyning_threaded.h file. The nano command was using to edit the file

\begin{Cpp}
 nano src/CHeterodyning_threaded.h
 for i in {1..5}; do make run; done
\end{Cpp}

where the thread count was changed in the following line to the values 1, 2, 4, 8, 16, and 32 in turn.

\begin{Cpp}
 #define Thread_Count 1
\end{Cpp}

The threaded C code was executed for each change of thread count using the following code

\begin{Cpp}
 make threaded
 for i in {1..5}; do make run_threaded; done
\end{Cpp}