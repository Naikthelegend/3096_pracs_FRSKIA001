\section{Methodology}

\subsection{Hardware}
The hardware used for this practical was a macOS laptop running Virtual Box with 4 CPU cores and 8 Gb of RAM. The Virtual Box was running Ubuntu image, which was in turn simulating a Raspberry Pi emulator. The Raspberry Pi emulator was used for the tests.

\subsection{Implementation}
The following code is the sudo code format of the two files: the python code which was used as the benchmark for the C code tests and the non-threaded C code.
The third file that was used in this test was a multi-threaded version of the C code file.
The \text{'data'} and \text{'carrier'} arrays are predefined arrays found in another tile

%Sudo
\begin{Cpp}
 # Load data arrays from external file
 Load data from external file

 # Define values
 c = carrier array
 d = data array
 result = empty array

 # Main function
 function main():
  Display "There are " + length of c + " samples"
  Display "using type " + type of first element in data array
  Call Timing.startlog()
  for i from 0 to length of c - 1:
  Append c[i] * d[i] to result array
  Call Timing.endlog()
\end{Cpp}

\subsection{Experiment Procedure}
The experiment start by benchmarking in Python using the following code.
All tests were run 5 times and then an average was taken:

\begin{Cpp}
 cd Python
 for i in {1..5}; do python3 PythonHeterodyning.py; done
\end{Cpp}

\newline
The C code was tested next starting with the non-threaded version:
\begin{Cpp}
 cd ../C
 make
 for i in {1..5}; do make run; done
\end{Cpp}

The threaded version of the C code was tested next, where each test changed the defined number of threads in the  \textit{CHeterodyning\textunderscore threaded.h} file.
The nano command was using to edit the file:

\begin{Cpp}
 nano src/CHeterodyning_threaded.h
 for i in {1..5}; do make run; done
\end{Cpp}

Where the thread count was changed in the following line to the values 1, 2, 4, 8, 16, and 32 in turn:

\begin{Cpp}
 #define Thread_Count 1
\end{Cpp}

The threaded C code was then executed for each change of thread count:

\begin{Cpp}
 make threaded
 for i in {1..5}; do make run_threaded; done
\end{Cpp}

\newline
The next test involved the use of compiler optimisation flags which optimisation which optimise the code for various application.
One of the following were using at a time:

\begin{itemize}
 \item -O0
 \item -O1
 \item -O2
 \item -O3
 \item -Ofast
 \item -Os
 \item -Og
\end{itemize}

They were added to the The \textit{makefile} under the \textit{\$(CFLAG)} heading and then the non-threaded version was run:

\begin{Cpp}
 nano makefile
\end{Cpp}

The \textit{-funroll-loops} flag unrolls loops into repetitive code.
It can be combined with any of the flags used in the previous test, however, for efficiencies sake, it was only applied to the slowest and fasted two compiler optimisation flags.

\newline
The last test involves altering bit widths of the data arrays.
This involves change the data type of the arrays in \textit{globals.h} and the output array in \textit{CHeterodyning\textunderscore threaded.h}
The data types that were tested were

\begin{itemize}
 \item float
 \item double
 \item \textunderscore \textunderscore fp16
\end{itemize}

The following code quickly changes the data types in the array where \textit{abc} and \textit{xyc} can be substituted as to the data type in the file and the data type to change to respectively

\begin{Cpp}
 sed -i -e 's/abc/xyz/g' src/globals.h
 sed -i -e 's/abc/xyz/g' src/CHeterodyning.c
\end{Cpp}

Importantly, when the data type \textit{\textunderscore \textunderscore fp16} is used, the compiler flag \textit{-mfp-format=ieee}.
All added compiler flags were removed before running this test, and the non-threaded version was used

\newline
Finally the best combination of the above number of threads, compiler flags including \textit{-funroll-loops} and bit width was used to generate the fastest C code execution.