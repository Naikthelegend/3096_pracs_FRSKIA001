\section{Methodology}

\subsection{Hardware}
The hardware used for this practical was a macOS laptop running Virtual Box with 4 CPU cores and 8 Gb of RAM. The Virtual Box was running Ubuntu image, which was in turn simulating a Raspberry Pi emulator. The Raspberry Pi emulator was used for the tests.

\subsection{Implementation}
The following code is the sudo code format of the three files: the python code which was used as the benchmark for the C code tests,

%Sudo
\begin{Cpp}
 # Load data arrays from external file
 Load data from external file

 # Define values
 c = carrier array
 d = data array
 result = empty array

 # Main function
 function main():
 Display "There are " + length of c + " samples"
 Display "using type " + type of first element in data array
 Call Timing.startlog()
 for i from 0 to length of c - 1:
 Append c[i] * d[i] to result array
 Call Timing.endlog()
 }
\end{Cpp}

\begin{Cpp}
 nano makefile
 make
 for i in {1..5}; do make run; done
\end{Cpp}

or you could turn it into a float: see listing~\ref{lst:OpenCL_Matrix_Mult}.  Floats are tables, figures and listings that appear at a different place than in the source code.  This template is set up to put floats at the top of the next column, as prescribed by the IEEE article specification.

\begin{OpenCL_float}{OpenCL kernel to perform matrix multiplication}{OpenCL_Matrix_Mult}
__kernel void Multiply(
 __global float* A, // Global input buffer
 __global float* B, // Global input buffer
 __global float* Y, // Global output buffer
   const  int    N  // Global uniform
){
 const int i = get_global_id(0); // 1st dimension index
 const int j = get_global_id(1); // 2nd dimension index

 // Private variables
 int   k;
 float f = 0.0;

 // Kernel body
 for(k = 0; k < N; k++) f += A[i*N + k] * B[k*N + j];
 Y[i*N + j] = f;
}
\end{OpenCL_float}

Only list what is relevant.  Don't give too much detail - just enough to show what you've done.  This template supports the following languages:

\begin{itemize}
 \item Matlab (Octave)
 \item GLSL
 \item OpenCL
 \item Verilog
 \item C++ (use the name ``Cpp'')
\end{itemize}
  
\subsection{Experiment Procedure}
Furthermore, include detail relating to the experiment itself: what did you do, in what order was this done, why was this done, etc.  What are you trying to prove / disprove?